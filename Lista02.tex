\documentclass[12pt,a4paper]{article}
\usepackage[portuguese]{babel}
\usepackage[utf8]{inputenc}
\usepackage{graphicx}
\usepackage{amsmath}
\begin{document}
\begin{titlepage}
	\centering
  \includegraphics[width=0.33\textwidth]{usp}\par\vspace{1cm}
  {\scshape\LARGE Universidade de São Paulo\par}
	\vspace{1cm}
  {\scshape\Large SEL0611- Fundamentos de Controle\par}
	\vspace{1.5cm}
  {\huge\bfseries Lista de Exercícios No.2\par}
	\vspace{2cm}
  {\Large\itshape Pedro Morello Abbud \par}
  \vspace{1cm}
  Número USP 8058718
	\vfill
  Disciplina minsitrada por\par
  Professor Doutor B.J.Mass

	\vfill 
% Bottom of the page
	{\large \today\par}
\end{titlepage}
\newpage
\section*{Exercícios}
\begin{enumerate}
  \item Identifique o livro apresentado, comentado e sugerido em sala para SEL0611 - 2016,
indicando por escrito o sobrenome dos autores (pelo menos o de um deles).

Sistemas de Controle Modernos, escrito por Richard C.\ Dorf e Robert H.\ Bishop.

\item Qual o aplicativo computacional apresentado e comentado em sala de aula, e que
sera empregado em SEL0611 - 2016, durante todo o semestre?

O aplicativo é o Matrix Lab (MATLAB).

\item Descreva suscintamente o que faz o comando ``tf" do aplicativo computacional
  referido no exercício 2 acima, e apresentado em sala de aula.

  O comando ``tf" cria uma função de transferência, real ou complexa, com os numerador 
  e denominador descritos pelos argumentos dados a função.

\item Qual o objeto matemático fundamental para SEL0611 discutido na primeira aula 
  do semestre na apresentacão de sistemas dinâmicos? 

  O objeto fundamental é chamado de \emph{Equação Diferencial Ordinária}.

\item O objeto fundamental que define um sistema dinâmico no contexto de \emph{Fundamentos
    de Controle} e pouco empregado na forma direta ou bruta; preferimos uma forma 
  alternativa, ja apresentada em \emph{Sinais e Sistemas} e possivelmente \emph{Circuitos Elétricos}.
Que forma alternativa é esta ? 

Esta forma alternativa é chamado de \emph{Função de transferência}.

\item Qual o conceito ou ideia fundamental que pode ser considerado quase equivalente
a de sistemas (ou sistemas) de controle?\\ Essa ideia foi apresentada e discutida em
sala de aula.

A ideia fundamental é o \emph{Diagrama de Blocos}.
\end{enumerate}
\end{document}
