\documentclass[12pt,a4paper]{article}
\usepackage[portuguese]{babel}
\usepackage[utf8]{inputenc}
\usepackage{graphicx}
\usepackage{amsmath}
\usepackage{enumitem}
\usepackage{mathrsfs}
\usepackage{amsfonts}
\usepackage[framed,numbered,autolinebreaks,useliterate]{mcode}
\usepackage[all,cmtip]{xy}
\newcommand{\fblock}[1]{*++[F]{#1}}
\begin{document}
\begin{titlepage}
  \centering
  \includegraphics[width=0.33\textwidth]{usp}\par\vspace{1cm}
  {\scshape\LARGE Universidade de São Paulo\par}
  \vspace{1cm}
  {\scshape\Large SEL0611- Fundamentos de Controle\par}
  \vspace{1.5cm}
  {\huge\bfseries Lista de Exercícios No.4\par}
  \vspace{2cm}
  {\Large\itshape Pedro Morello Abbud \par}
  \vspace{1cm}
  Número USP 8058718
  \vfill
  Disciplina minsitrada por\par
  Professor Doutor B.J.Mass

  \vfill 
  % Bottom of the page
  {\large \today\par}
\end{titlepage}
\newpage
\section*{Exercícios}
\begin{enumerate}
  \item  Um sistema cujo comportamento é definido ou especificado por uma equação 
    diferencial ordinária é chamado ”sistema dinâmico”. Um sistema dinamico com uma ˆ
entrada $u(t)$ e uma saída $y(t)$ pode ser representado graficamente por um diagrama
de blocos, caso nos interessemos apenas pelo comportamento da entrada, da saída,
e da relação entre as mesmas.
\begin{enumerate}[label=(\alph*)]
    \begin{figure}
    \[
\xymatrix{
  \ar[r]^{Input}& *++[F]{EDO}\ar[r]   
}
\]
\end{figure}
  \item Verifique se a figura acima
\end{enumerate}
\end{enumerate}
\end{document}
