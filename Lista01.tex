\documentclass[12pt,a4paper]{article}
\usepackage[portuguese]{babel}
\usepackage[utf8]{inputenc}
\usepackage{graphicx}
\usepackage{amsmath}
\usepackage{enumitem}
\usepackage{mathrsfs}
\usepackage{amsfonts}
\usepackage[framed,numbered,autolinebreaks,useliterate]{mcode}
\begin{document}
\begin{titlepage}
  \centering
  \includegraphics[width=0.33\textwidth]{usp}\par\vspace{1cm}
  {\scshape\LARGE Universidade de São Paulo\par}
  \vspace{1cm}
  {\scshape\Large SEL0611- Fundamentos de Controle\par}
  \vspace{1.5cm}
  {\huge\bfseries Lista de Exercícios No.1\par}
  \vspace{2cm}
  {\Large\itshape Pedro Morello Abbud \par}
  \vspace{1cm}
  Número USP 8058718
  \vfill
  Disciplina minsitrada por\par
  Professor Doutor B.J.Mass

  \vfill 
  % Bottom of the page
  {\large \today\par}
\end{titlepage}
\newpage
\section*{Exercícios}
\begin{enumerate}
  \item  Um sistema cujo comportamento é definido ou especificado por uma equação 
    diferencial ordinária é chamado ``sistema dinâmico". Um sistema dinâmico com
    uma entrada $u(t)$ e uma saída $y(t)$ pode ser representado graficamente por
    um diagrama de blocos, caso nos interessemos apenas pelo comportamento da 
    entrada, da saída, e da relação entre as mesmas.
\begin{enumerate}[label=(\alph*)]
    \begin{figure}
      \label{fig:1}
    \end{figure}
  \item Verifique se a figura acima contém informações relevantes.
  \item O que poderia ser introduzido no lugar da Figura~\ref{fig:1}, no lugar 
    da sigla E.D.O?
\end{enumerate}

\emph{Resolução:}

\begin{enumerate}[label=(\alph*)]
  \item A figura~\ref{fig:1} nos mostra que em sistemas dinâmicos podemos 
    correlacionar a entrada e a saída de um sistema dinâmico através de uma 
    Equação Diferencial Ordinária. A fim de facilitar a visualização do 
    dessa correlação, podemos usar a ferramenta gráfica chamada Diagrama de 
    Blocos.
  \item Podemos introduzir no lugar da sigla Equação Diferencial Ordinária, uma 
    função de transferência, pois relaciona um conjunto de dados ou informações 
    de entrada e os modifica em um novo conjunto de dados na saída. Quando temos 
    sistemas casuais e  lineares invariantes no tempo, a função de transferência
    relaciona a entrada e a saída através da transformada de Laplace.
\end{enumerate}
\item Comprove que um sistema com entrada $u(t)$ e saída $y(t)$, cujo comportamento
  dinâmico é descrito por uma equação diferencial da forma:
  \begin{equation}
    \label{eq:1}
    \frac{d^{n}y}{dt^{n}}+a_{n-1}\frac{d^{n-1}y}{dt^{n-1}}+\cdots+a_1\frac{dy}{dt} =
    u(t)
  \end{equation}
  tem uma função de transferência na forma:
  \begin{equation}
    \label{eq:2}
    \frac{Y(s)}{U(s)}=\frac{1}{s^n + a_{n-1}+\cdots+a_1s+a_o}
  \end{equation}
  Obs: Uma função de transferência só é definida para condições inicias nulas, 
  i.e.\,\ para:
  \begin{equation}
    \label{eq:3}
    y^{(0)}(0)=y^{(1)}(0)=\cdots=y^{(n-1)}(0)=0
  \end{equation}

  \emph{Resolução}

  Se sabemos que as transformadas de Laplace abaixo são verdadeiras:
  \begin{align}
    \label{derivada}
\mathcal{L}\{y'\}
  &= s \mathcal{L}\{y\} - y(0) \\
  \mathcal{L}\{y''\}
  &= s^2 \mathcal{L}\{y\} - s y(0) - y'(0) \\
  \begin{split} 
  \mathcal{L}\left\{ y^{(n)} \right\}
&=  s^n \mathcal{L}\{y\} - s^{n - 1} y(0) - s^{n - 2} y'(0)- \cdots - y^{(n - 1)}(0)\\
  &= s^n\mathcal{L}\{y\} - \sum_{k=0}^{n-1}(s^{(n-1-k)}y^{(k)}(0))
\end{split}
\end{align}
E que a transformada de Laplace é linear como descrito abaixo:
\begin{align}
  \mathcal{L}\left\{a f(t) + b g(t) \right\}
 &= a \mathcal{L}\left\{ f(t) \right\} +
    b \mathcal{L}\left\{ g(t) \right\}
\end{align}
É fácil inferir que:
\begin{align}
  \mathcal{L}\left\{\frac{d^{n}y}{dt^{n}}+a_{n-1}\frac{d^{n-1}y}{dt^{n-1}}+
    \cdots+a_1\frac{dy}{dt}+a_{0}\right\} & = \mathcal{L}\left\{u(t)\right\}  \\
=  Y(s)\mathcal{L}\left\{\frac{d^{n}}{dt^{n}}+a_{n-1}\frac{d^{n-1}}{dt^{n-1}}+
  \cdots+a_1\frac{d}{dt}+a_{0}\right\}&=U(s)\\
  \begin{split}
    =s^n - \sum_{k=0}^{n-1}(s^{(n-1-k)}y^{(k)}(0))+\cdots\\+a_{2}\left( s^2 - s y(0) - y'(0) \right)+a_{1}\left(s - y(0)\right)+a_0&=\frac{U(s)}{Y(s)}
\end{split}
\end{align}
Considerando todas as condições inicias zero, e tomando o inverso de cada lado, 
temos novamente a equação (\ref{eq:2}):
\begin{align*}
    \frac{Y(s)}{U(s)}=\frac{1}{s^n + a_{n-1}+\cdots+a_1s+a_o} 
  \end{align*}
como queríamos demonstrar.
\item Para o sistema do exercício anterior verifique que a razão $Y(s)/U(s)$ 
  não será independente de $U(s)$ quando as condições iniciais não forem todas nulas.

  Tal observação pode ser notada ao observar a propriedade básica que relaciona a 
  derivada de uma função no tempo com seu equivalente em frequência, mostrados 
  na equação~\ref{derivada}, enxergamos ali que se a condição inicial não for 
  nula, não poderemos manipular o termo $Y(s)$ de forma independente, pois qualquer 
  constante não-nula dada pela condição inicial não é de fato 
  multiplicada por esse termo.
\item Escreva um trecho de \emph{MATLAB} que declare os polinômios do numerador e do 
  denominador, e que determine e retorne a função de transferência abaixo.\\
  Escolha nomes que lhe convier e empregue o comando ``tf ” discutido em 
  sala-de-aula.
  \begin{equation}
    \label{eq:4}
    H(s)=\frac{1}{s^2 +0,2 s + 1}
  \end{equation}

  \emph{Resolução}
     \begin{lstlisting}[label=matlab:1, caption=Código em MATLAB para a função de transferência $g_1(s)$ associada a equação~\ref{eq:4}]
      num=1 
      denom= [1 0.2 5]
      g1=tf(num,denom)
    \end{lstlisting}

\item Dada a equação diferencial ordinária (EDO): 
  \begin{equation}
    \label{eq:5}
    y''(t)+2y'(t)+5y(t)= 10x(t)
  \end{equation}
  onde $y(t)$ e $x(t)$ são respectivamente a saída e a entrada de um circuito; 
  determine a função de transferência correspondente. 
  
  \emph{Resolução}

  Aplicando a transformando de Laplace em ambos os lados, obtemos:

\begin{align}
  \mathcal{L}\left\{y''(t)+2y'(t)+5y(t)\right\}=\mathcal{L}\left\{10x(t)\right\}\\
  =s^2Y(S)+2sY(S)+5Y(S)=10 X(S)\\
  =\frac{Y(S)}{X(S)}=\frac{1}{s^2+2s+5}
  \end{align}
  Que é a função de transferência correspondente do circuito.
\item Escolha nomes apropriados (letra minúscula é melhor) e escreva um trecho de 
  \emph{MATLAB} para declarar os polinômios do numerador e do denominador, e produzir 
  a função de transferência associada a equação (\ref{eq:4}).

  \emph{Resolução}

     \begin{lstlisting}[label=matlab:2, caption=Código em MATLAB para a função de transferência $g_1(s)$ associada a equação~\ref{eq:4}]
      g1=tf(1,[1 0.2 1]) 
    \end{lstlisting}
 
\end{enumerate}
\end{document}
