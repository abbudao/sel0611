\documentclass[12pt,a4paper]{article}
\usepackage[portuguese]{babel}
\usepackage[utf8]{inputenc}
\usepackage{graphicx}
\usepackage{amsmath}
\usepackage{enumitem}
\usepackage{mathrsfs}
\usepackage{amsfonts}
\usepackage[framed,numbered,autolinebreaks,useliterate]{mcode}
\begin{document}
\begin{titlepage}
  \centering
  \includegraphics[width=0.33\textwidth]{usp}\par\vspace{1cm}
  {\scshape\LARGE Universidade de São Paulo\par}
  \vspace{1cm}
  {\scshape\Large SEL0611- Fundamentos de Controle\par}
  \vspace{1.5cm}
  {\huge\bfseries Lista de Exercícios No.4\par}
  \vspace{2cm}
  {\Large\itshape Pedro Morello Abbud \par}
  \vspace{1cm}
  Número USP 8058718
  \vfill
  Disciplina minsitrada por\par
  Professor Doutor B.J.Mass

  \vfill 
  % Bottom of the page
  {\large \today\par}
\end{titlepage}
\newpage
\section*{Exercícios}
\begin{enumerate}
  \item Considere um sistema com uma única entrada $u(t)$ e uma única saída $y(t)$, 
    cujo comportamento dinâmico é descrito por uma equação diferencial ordinária 
    (EDO) da forma:

    \begin{align}
      \label{edo:1}
      \frac{d^n y}{dt^n}+b_{n-1}\frac{d^{n-1} y}{dt^{n-l}}+\cdots+b_1 \frac{dy}{dt}+b_0 y = u(t)
    \end{align}
\begin{enumerate}[label=(\alph*)]
  \item Convença-se de que para um sistema assim o modelo dinâmico mais natural 
é a própria EDO (\ref{edo:1}).
\item Em complexidade o nível de modelagem seguinte e a função de transferência 
(FT) correspondente. Verifique que para este caso o modelo corresponde à:

\begin{align}
  \label{edo:2}
  \frac{Y(S)}{U(S)}= \frac{1}{s^n + b_{n-1}s^{n-1}+\cdots+b_1 s+ b_0}
\end{align}

  \end{enumerate}

\begin{enumerate}[label=(\alph*)]
  \item Sabemos que 
  \item Da equação diferencial ordinária (\ref{edo:1}) podemos induzir:
    \begin{align*}
      \frac{1}{\frac{d^n}{dt^n}+b_{n-1}\frac{d^{n-1} }{dt^{n-l}}+\cdots+b_1 \frac{d}{dt}+b_0}  &= \frac{y}{u(t)}
    \end{align*}
    Aplicando a transformada de Laplace em ambos os lados da igualdade:
    \begin{align*}
      \mathbb{L}\left\{ \frac{1}{\frac{d^n}{dt^n}+\frac{d^{n-1}}{dt^{n-l}}+\cdots+b_1 \frac{d}{dt}+b_0} \right\}  &= \frac{Y(S)}{U(S)}
    \end{align*}
    Como sabemos que $\mathbb{L}\left\{\frac{d^n}{dt^n}\right\}=s^n$, conseguimos então
    deduzir a equação diferencial ordinária de número (\ref{edo:2}).
\end{enumerate}
\item Obtenha o modelo EDO de um sistema cujo modelo TF e simplesmente:
  \begin{align}
    \label{edo:3}
    H(S)=\frac{1}{s^2 + 0,2s+1}
  \end{align}
  Podemos obter o modelo da EDO trocando o domínio da função de transferência da
  frequência para o tempo:

\item Para um circuito eletrico descrito por:
  \begin{align}
    \frac{d^2 y}{dt^2}+2\frac{dy}{dt}+5y(t)= 10x(t)
  \end{align} 
    onde a entrada é $x(t)$ e a saída é $y(t)$, o modelo EDO é uma única equação de $2^a$ ordem.
    Por outro lado, o modelo EDO de um motor CC por exemplo, é normalmente um par de equações de $1^a$ ordem: 
    \begin{align}
      \label{edo:4}
    \begin{cases}
      v= Ri+ L\frac{di}{dt}+K_E \omega \\
      J\frac{d \omega}{dt}=K_T i - K_B \omega 
    \end{cases}
  \end{align}
  No caso geral e sempre possível transformar uma EDO de ordem $n$, em $n$ EDOs de
  ordem ``um'' cada uma. Verifique se e possível reescrever o modelo da EDO representado
  em (\ref{edo:4}), na forma de duas EDOs de primeira ordem.

\end{enumerate}
\end{document}
